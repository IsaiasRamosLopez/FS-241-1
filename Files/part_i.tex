\begin{center}
	\underline{\Large\scshape\bfseries \dytema}
\end{center}
%\markright{INFORME LABORATORIO Nº 05}
%-------------------------------------------------------------------------------------------
\section{Objetivos}
\begin{enumerate}[label=\itemcirccz{azzul}{\arabic*},itemsep=2pt,partopsep=6pt]
	\item Conocer las formas de electrización.
	\item Explicar en qeu consiste las formas de electrización.
	\item Manejar los instrumentos.
\end{enumerate}

%==========================================================================================
\section{Fundamento teórico}
% La cantidad de calor recibido o cedido por un cuerpo se calcula mediante la
% siguiente fórmula.
% \[Q=m\cdot c\cdot(T_f-T_i)\]
% \sdconditions[12cm]{azzul}{Donde:}{%
% 	\begin{tabular}{lcl}
% 		$Q$   & \@: & Cantidad de Calor.   \\
% 		$m$   & \@: & Masa.                \\
% 		$c$   & \@: & Calor Específico.    \\
% 		$T_f$ & \@: & Temperatura Final.   \\
% 		$T_i$ & \@: & Temperatura Inicial.
% 	\end{tabular}
% }
% \begin{itemize}[label=\textbf{$\bullet$},itemsep=2pt,partopsep=6pt]
% 	\item $m_v$ es la masa del vaso del calorímetro y $c_v$ calor específico.
% 	\item $m_t$ la masa de la parte sumergida del termómetro y $c_t$ calor específico.
% 	\item $m_a$ la masa de la parte sumergida del ambiente  y $c_a$ calor específico.
% \end{itemize}
%==========================================================================================
\section{Equipos y Materiales}
\begin{enumerate}[label=\itemcirccz{azzul}{\alph*},itemsep=2pt,partopsep=6pt]
	\item Franela.
	\item Tubo de vidrio.
	\item {\color{morado01} New Item.}
	\item {\color{morado01} New Item.}
	\item {\color{morado01} New Item.}
\end{enumerate}

% \begin{multicols}{2}
% 	\setlength{\columnseprule}{0pt}
% 	\begin{figure}[H]
% 		\centering
% 		%\caption{Dinamómetro}
% 		\includegraphics[width=2.5cm]{Images/Image_iii.jpeg}
% 	\end{figure}

% 	\begin{figure}[H]
% 		\centering
% 		\includegraphics[width=7cm]{Images/Image_iv.jpeg}
% 		\caption{Vaso precipitado}
% 	\end{figure}

% 	\begin{figure}[H]
% 		\centering
% 		\includegraphics[width=7.5cm]{Images/Image_v.jpeg}
% 		\caption{Calorímetro.}
% 	\end{figure}
% \end{multicols}
%==========================================================================================
\section{Procedimiento Experimental}
% \begin{dyNoteImportant}[morado01!20]{azulfor!10}{black!80}{Procedimientos}
% 	\begin{itemize}[label=\textbf{$\bullet$},itemsep=2pt,partopsep=6pt]
% 		\item Arme el equipo como se muestra en la figura.
% 		\item Calienta la miuestra de masa $m$ y calor específico $c$ a $T_1$, sumergido en
% 		      el agua en ebullición.
% 		\item Sumerge la muestra en agua fría de masa $M$ que contiene un calorímetro de masa
% 		      $M'$ y de calor específico $c'$.
% 	\end{itemize}
% \end{dyNoteImportant}
% \subsection{Para el Agua}
% \begin{enumerate}[label=\bfseries\alph*.-,itemsep=2pt, partopsep=6pt]
% 	\item \textbf{Calculando el equivalente en agua del calorímetro}
% 	      \[k=\cfrac{m(T-T_e)}{T_e-T_0}-M\]
% 	      Reemplzando:
% 	      \[k=\cfrac{200(91-54)}{54-18.5}-120\]
% 	      Finalmente:
% 	      \[k=88.54\]
% 	\item \textbf{Calculando el calor específico del sólido}
% 	      \[c=\cfrac{(M+k)(T_e-T_0)}{m\cdot(T-T_e)}\]
% 	      Reemplzando:
% 	      \[c=\cfrac{(120+k)(54-18.5)}{200\times(91-54)}\]
% 	      Finalmente:
% 	      \[c=1.000\,\,4\]
% \end{enumerate}
%==========================================================================================
\section{Procedimiento de Datos Experimentados}
%==========================================================================================
\section{Conclusiones}
%==========================================================================================
%-------------------------------------------------------------------------------------------
% \begin{table}[H]
% 	\centering
% 	\begin{tabular}{L{2cm}C{3cm}C{3cm}}
% 		\rowcolor{gray!10}Muestra & Valor Teórico & Valor Esperimental \\
% 		Agua                      & $1$           & $1.0004$           \\
% 		Aluminio                  & $0.11$        & $0.12$             \\
% 		Hierro                    & $0.22$        & $0.23$             \\
% 		Cobre                     & $0.093$       & $0.09$             \\
% 	\end{tabular}
% \end{table}
%==========================================================================================